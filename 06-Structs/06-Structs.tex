\documentclass{beamer}

\usepackage[italian]{babel}
\usepackage{minted}
\usepackage{mdframed}
\usepackage{hyperref}

\surroundwithmdframed{minted}

\newminted{cpp}{linenos,autogobble,breaklines}

\newcommand{\scaledimage}[1]{\includegraphics[width=\textwidth,height=0.8\textheight,keepaspectratio]{#1}}

\usetheme{Boadilla}
\title{06 - Structs}
\author{Bailetti Tommaso}
\institute{ITI Don Orione}
\date{10 Marzo 2022}

\begin{document}
    \begin{frame}
        \titlepage
    \end{frame}
    
    \begin{frame}
        \frametitle{I dati non omogenei sono fastidiosi.}
    
        Ormai (dovremmo) aver imparato quali sono i tipi di variabile utilizzabili in C++. Tuttavia questi non si prestano efficentemente quando dobbiamo utilizzare molti dati di diverso tipo.

        \onslide<2->
        Banalmente abbiamo visto l'esempio del registro, comprendiamo che non è una cosa facile gestire i diversi tipi di dati, anzi, diventa un problema a ogni tipo di dato che aggiungiamo.
    
    \end{frame}

    \begin{frame}[fragile]
        \frametitle{Cosa sono le \mintinline{cpp}{struct}?}
    
        Le \mintinline{cpp}{struct} sono un tipo di dato che definisce il programmatore, il quale contiene al suo interno variabili non necessariamente omogenee.

        \onslide<2->
        Di fatto creiamo un nuovo tipo di variabile, che possiamo usare all'interno della nostra applicazione, vediamo come:

        \onslide<3->
        \begin{cppcode}
            struct user {
                string name;
                string surname;
                int birthDay;
                int birthMonth;
                int birthYear;
            };
        \end{cppcode}
    
    \end{frame}

    \begin{frame}[fragile]
        \frametitle{Okay, tutto bellissimo, ma come si usa?}
    
        Non ci sono di fatto cose particolari da scrivere per poter usare una \mintinline{cpp}{struct}, posso dichiarare una variabile di tipo struttura con il codice seguente:

        \onslide<2->
        \begin{cppcode}
            user coolUser;
        \end{cppcode}
    
    \end{frame}

    \begin{frame}[fragile]
        \frametitle{Stupendo, ma per le variabili "dentro"?}
    
        Per le variabili poste all'interno della nostra struttura, possiamo assegnarle o recuperarle tramite la seguente sintassi:
        
        \onslide<2->
        \begin{cppcode}
            struct user {
                string name;
                string surname;
                int birthDay;
                int birthMonth;
                int birthYear;
            };
            user coolUser;
            // tirerà fuori il nome dell'utente
            cout << coolUser.name;
            // inserirà "Bailetti" dentro surname
            coolUser.surname = "Bailetti";
        \end{cppcode}
    
    \end{frame}

    \begin{frame}[fragile]
        \frametitle{Inizializzazione di una \mintinline{cpp}{struct}}
    
        Come inizializziamo, quando dichiariamo una variabile di tipo \mintinline{cpp}{struct}, la nostra variabile?

        \onslide<2->
        Utilizziamo la stesso modo per inizializzare gli array:
        \begin{cppcode}
            // attenzione che è necessario che quello che inserite sia dello stesso TIPO
            user coolUser = {"Tommaso", "Bailetti", 1, 4, 2004};
        \end{cppcode}
    
    \end{frame}

    \begin{frame}[fragile]
        \frametitle{Operazioni con le \mintinline{cpp}{struct}}
    
        Possiamo, come abbiamo visto prima, inserire o ottenere i valori dalla nostra struttura tramite il "punto", ma non solo.
        
        \onslide<2->
        Ad esempio, come copio tutti i valori da una variabile \mintinline{cpp}{struct} a un'altra?
        
        \onslide<3->
        \begin{cppcode}
            user coolUser;
            user anotherCoolUser;
            // questo copia tutti i campi da coolUser a anotherCoolUser
            anotherCoolUser = coolUser;
        \end{cppcode}
    
    \end{frame}

    \begin{frame}[fragile]
        \frametitle{Posso inserire \mintinline{cpp}{struct} dentro \mintinline{cpp}{struct}?}
        
        \pause
        Certo, ecco come:
        \begin{cppcode}
            struct user {
                string name;
                string surname;
                struct dataDiNascita {
                    int birthDay;
                    int birthMonth;
                    int birthYear;
                } 
            }
        \end{cppcode}
    
    \end{frame}

    \begin{frame}[fragile]
        \frametitle{E gli array? Funzionano allo stesso modo?}
    
        \pause
        Si, bisogna giusto prestare attenzione a come accedere agli stessi:
        \begin{cppcode}
            // dichiaro la struttura
            struct anotherUser {
                string name;
                int grades[10] = {};
            };
            // dichiaro e inizializzo la variabile
            anotherUser user = {};
            // inserisco il primo voto
            user.grades[0] = 8;  
        \end{cppcode}
    
    \end{frame}
\end{document}