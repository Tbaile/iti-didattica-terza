\documentclass{beamer}

\usepackage[italian]{babel}
\usepackage{minted}
\usepackage{mdframed}

\surroundwithmdframed{minted}

\newminted{cpp}{linenos,autogobble,breaklines}

\newcommand{\scaledimage}[1]{\includegraphics[width=\textwidth,height=0.8\textheight,keepaspectratio]{#1}}

\usetheme{Boadilla}
\title{04 - Arrays}
\author{Bailetti Tommaso}
\institute{ITI Don Orione}
\date{17 febbraio 2022}

\begin{document}

    \begin{frame}
        \titlepage
    \end{frame}

    \begin{frame}
        \frametitle{Definizione di Array}
        Gli array sono un insieme omogeneo di variabili, le quali si possono riferire come un unico insieme.
        \onslide<2->
        I singoli elementi si identificano con un indice.
    \end{frame}

    \begin{frame}[fragile]
        \frametitle{Vettori}
        Un vettore non è altro che una variabile che referenzia multiple variabili dello stesso tipo. Posso istanziare un vettore con la seguente sintassi:
        \onslide<2->
        \begin{cppcode}
            float alfa[10];
        \end{cppcode}
    \end{frame}

    \begin{frame}
        \frametitle{Vettori}
        Analizziamo cosa abbiamo di fatto scritto nel codice prima:
        \begin{description}
            \item[float]<2-> è il tipo del nostro array
            \item[alfa]<3-> è il nome del nostro array
            \item[10]<4-> è il numero di valori che vogliamo salvare all'interno dell'array.  
        \end{description}
        \begin{block}<5->{Attenzione}
            Non è possibile cambiare quanti valori salviamo all'interno dell'array.
        \end{block}
    \end{frame}

    \begin{frame}[fragile]
        \frametitle{Definizione di un array}
    
        Gli array sono definibili in diversi modi, alcuni modi li trovate sotto:
        \onslide<2->
        \begin{cppcode}
            int alpha[] = {0, 1, 2, 3, 4};
            int beta[5] = {1, 2, 3, 4, 5};
            //Inizializzo tutti i valori a 0
            int gamma[10] = {}; 
        \end{cppcode}
        \begin{block}<3->{Nota}
            Utilizzare "\mintinline{cpp}{= {0}}" inizializza tutto l'array a 0, tuttavia se utilizziamo ad esempio "\mintinline{cpp}{= {10}}" \textbf{SOLO} il primo elemento viene inizializzato a 10, gli altri rimarranno a 0.
        \end{block}
    
    \end{frame}
    
    \begin{frame}[fragile]
        \frametitle{Utilizzare un array}
            
        Un array si usa come una variabile, tuttavia va inserito tra le partentesi quadre l'indice dell'elemento che voglio utilizzare.
        \onslide<2->
        \begin{cppcode}
            int alpha[] = {10, 20, 30, 40, 50};
            // Inserisco 100 al posto di 20
            alpha[1] = 100;
            // Stamperà "30" a console
            cout << alpha[2];
            // L'input dell'utente sovrascriverà il valore "50"
            cin >> alpha[4];
        \end{cppcode}
    
    \end{frame}

    \begin{frame}[fragile]
        \frametitle{Iterare su vettori}

        Iterare un vettore è sostanzialmente molto semplice, è necessario utilizzare un ciclo \mintinline{cpp}{for} e conoscere la dimensione dell'array.
        \onslide<2->
        \begin{cppcode}
            const int DIM_ARRAY = 10;

            int data[DIM_ARRAY] = {};
            
            for (int i = 0; i < DIM_ARRAY; i++) {
                data[i] = 10*(i + 1);
            }
            
            for (int i = 0; i < DIM_ARRAY; i++) {
                cout << i + 1 << "° valore è: " << data[i] << endl;
            }
        \end{cppcode} 

    \end{frame}

    \begin{frame}[fragile]
        \frametitle{Matrici di valori}
    
        C'è la possibilità di aggiungere ai nostri vettori un'ulteriore dimensione. 
        \onslide<2->

        Vedetela come se con gli array a singola dimensione abbiamo dei vettori, qui abbiamo delle tabelle.
        \onslide<3->

        \begin{cppcode}
            const int SQUARED_SIZE = 5;

            int table[SQUARED_SIZE][SQUARED_SIZE] = {};

            for (int i = 0; i < SQUARED_SIZE; i++) {
                for (int j = 0; j < SQUARED_SIZE; j++) {
                    cout << table[i][j] << "\t";
                }
                cout << endl;
            }
        \end{cppcode}
    
    \end{frame}

    \begin{frame}
        \frametitle{Gli array e le funzioni}
    
        È possibile utilizzare gli array come parametri delle funzioni, \textbf{bisogna stare attenti} perché gli array passati come parametri sono \textbf{SEMPRE} passati per riferimento.
    
    \end{frame}

    \begin{frame}[fragile]
        \frametitle{Gli array e le funzioni}
    
        \begin{cppcode}
            void printArray(int data[], int dimArray) {
                for (int i = 0; i < dimArray; i++) {
                    cout << data[i] << endl;
                }
            }

            int main() {
                const int ARRAY_DIM = 10;
                int data[ARRAY_DIM] = {};
                printArray(data, ARRAY_DIM);
                return 0;
            }
        \end{cppcode}
    
    \end{frame}

    \begin{frame}[fragile]
        \frametitle{Le stringhe}

        Una stringa non è altro che una serie di caratteri, di fatto è un vettore di \mintinline{cpp}{char}.
        
        \begin{cppcode}
            char cat[] = "gatto";
            char catChar[] = {'g', 'a', 't', 't', 'o', '\0'};
        \end{cppcode}

    \end{frame}

    \begin{frame}[fragile]
        \frametitle{Lunghezza di una stringa}
    
        Al contrario degli array, mi è possibile andare a indagare sulla lunghezza di una stringa. Basta che itero il mio vettore fino a trovare il carattere \mintinline{cpp}{'\0'}, ovvero basta trovare il terminatore di stringa.
        \onslide<2->
        \begin{cppcode}
            char cat[] = "gatto";
            int i = 0;
            while (cat[i] != '\0') {
                i++;
            }
            cout << "Dimensione stringa cat: " << i << endl;
        \end{cppcode}
    
    \end{frame}
\end{document}