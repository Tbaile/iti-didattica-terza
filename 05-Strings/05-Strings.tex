\documentclass{beamer}

\usepackage[italian]{babel}
\usepackage{minted}
\usepackage{mdframed}
\usepackage{hyperref}

\surroundwithmdframed{minted}

\newminted{cpp}{linenos,autogobble,breaklines}

\newcommand{\scaledimage}[1]{\includegraphics[width=\textwidth,height=0.8\textheight,keepaspectratio]{#1}}

\usetheme{Boadilla}
\title{05 - String}
\author{Bailetti Tommaso}
\institute{ITI Don Orione}
\date{03 Marzo 2022}

\begin{document}
    \begin{frame}
        \titlepage
    \end{frame}
    
    \begin{frame}
        \frametitle{Cosa è una stringa?}
    
        Abbiamo definito la stringa come un array di caratteri. Sappiamo come lavorare con un array, abbiamo visto come sostituire dei caratteri, abbiamo visto come contare i caratteri di una stringa, ecc...

        \onslide<2->
        Se vi dicessi che c'è un modo per semplificare tutte queste operazioni? (che come sappiamo, necessitano di cicli, di eventuali separazioni in funzioni per non ripetere il codice troppe volte, ecc...)
    
    \end{frame}

    \begin{frame}[fragile]
        \frametitle{L'oggetto \mintinline{cpp}{string}}
    
        Cambiamo dunque la definizione di "stringa".
        
        Vi presento l'oggetto \mintinline{cpp}{string}.

        \onslide<2->
        Esatto, incominceremo attivamente a usare un oggetto, cosa che in realtà facevamo già con \mintinline{cpp}{cin} e \mintinline{cpp}{cout}.
    \end{frame}

    \begin{frame}[fragile]
        \frametitle{Definizione e inizializzazione di \mintinline{cpp}{string}}

        Incominciamo con come definire e inizializzare una stringa. Vediamo le differenze tra array di caratteri e una stringa.

        \onslide<2->
        C'è già una cosa interessante. Con \mintinline{cpp}{string} noi ci dimentichiamo completamente della dimensione dell'array di caratteri.
        \onslide<3->
        \begin{cppcode}
            // con la prima variabile siamo limitati a 12 caratteri.
            char charHelloWorld[] = "Hello World!";
            string stringHelloWorld = "Hello World!;
        \end{cppcode}
    
    \end{frame}

    \begin{frame}[fragile]
        \frametitle{Concatenazione di stringhe}
    
        \begin{cppcode}
            string ab = "ab";
            string cd = "cd";
            string abcd = ab + cd;
            // sotto stamperà "abcd"
            // sto concatenando le due stringhe
            cout << abcd;
            // posso anche fare la concatenazione in maniera dinamica e con più elementi, sotto stamperà "ab cd"
            cout << ab + " " + cd;
        \end{cppcode}
    
    \end{frame}

    \begin{frame}[fragile]
        \frametitle{Lunghezza di una stringa}

        \begin{cppcode}
            string testoLungo = "abcdefghijklmnopqrstuvwxyz";
            // i due modi sotto sono equivalenti e stampano lo stesso numero
            cout << testoLungo.length() << endl;
            cout << testoLungo.size() << endl;
        \end{cppcode}

    \end{frame}

    \begin{frame}
        \frametitle{Metodi per svolgere tutto}
    
        Nonostante siamo in grado di accedere alla stringa come se fosse un array, è opportuno (ove possibile) utilizzare i \textbf{metodi} forniti dalla classe.

        Troverete molte più info a questo link: \url{https://www.cplusplus.com/reference/string/string/} e successivamente sulla repository dei laboratori.
    
    \end{frame}
\end{document}